%!TEX root = ../dissertation.tex

\begin{otherlanguage}{english}
\begin{abstract}

In the domains of Artificial Intelligence (AI), Machine Learning (ML), Big Data and video and image processing, the demand for faster computational systems has increased significantly. Traditional methods to improve performance, such as increasing computational cores and clock speed, often yield diminishing returns relative to the associated escalation in energy consumption and heat generation. 

This thesis emphasizes the importance of process efficiency within computational systems, specifically focusing on the impact of data fetching and manipulation on overall performance. Strategies like prefetching, data streaming, and vector processing emerge as potential avenues for improving throughput, energy efficiency, and power efficiency. The study aims to quantitatively evaluate the efficacy of incorporating streaming engines and SIMD operations into a CPU, providing insights into the viability and advantages of adopting these techniques for enhanced computational capabilities.


% Keywords
\begin{flushleft}

\keywords{Data Streaming, Scalable SIMD/Vector Processing, Data Manipulation Processor Microarchitecture, High-Performance Computing, CVA6, RISC-V.}

\end{flushleft}



\end{abstract}
\end{otherlanguage}